%%%%%%%%%%%%%%%%%%%%%%%%%%%%%%%%%%%%%%%%%%%%%%%%%%%%%%%%%%%%%%%%%%%%%%%%
%                                                                      %
%     File: Thesis_Conclusions.tex                                     %
%     Tex Master: Thesis.tex                                           %
%                                                                      %
%     Author: Andre C. Marta                                           %
%     Last modified :  2 Jul 2015                                      %
%                                                                      %
%%%%%%%%%%%%%%%%%%%%%%%%%%%%%%%%%%%%%%%%%%%%%%%%%%%%%%%%%%%%%%%%%%%%%%%%

\chapter{Conclusions}
\label{chapter:conc}

This chapter summarizes the relevant achievements of the work carried out in this thesis.

On Chapter \ref{chapter:1}, in the view of the case when a firm wants to enter the market with a new product, we conclude that:
\begin{itemize}
	\item In the benchmark model, the demand value that triggers the investment increases with the interest rate $r$, the market's uncertainty $\sigma$, the production capacity $K$ and the parameter $\alpha$; it decreases with the innovation level $\theta$ and the drift value $\mu$;
	\item In the capacity optimization model, the demand value that triggers the investment increases with the interest rate $r$, the market's uncertainty $\sigma$ and the parameter $\delta$; it decreases with the innovation level $\theta$; it has a non-monotonic behaviour with the drift value $\mu$.
	Moreover, the optimal capacity level increases with the drift value $\mu$; decreases with the interest rate $r$ and the parameter $\alpha$.
\end{itemize}

On Chapter \ref{chapter:2}, considering the case when an active firm wants to enter the market with a new product, replacing the old one, we conclude that:
\begin{itemize}
	\item In the benchmark model, the demand value that triggers the investment increases with the parameter $\delta$, the market's uncertainty $\sigma$ for $d_1 \in \left(1, \frac{1}{2} (3+\sqrt{5}) \right)$ and the parameter $\alpha$ for $\theta < \frac{K_1}{K_0^2}(K_0+K_1 r \delta)$; it decreases with the innovation level $\theta$, the market's uncertainty $\sigma$ for $d_1 >  \frac{1}{2} (3+\sqrt{5}) $ and the parameter $\alpha$ for $\theta > \frac{K_1}{K_0^2}(K_0+K_1 r \delta)$; it has a non-monotonic behaviour with the capacity of the old and new products $K_0$ and $K_1$, respectively, and the interest rate $r$.
	
	\item In the capacity optimization model, the demand value that triggers the investment increases with the parameter $\delta$; it decreases asymptotically with the innovation level $\theta$; it has a non-monotonic behaviour with the capacity of the old product $K_0$, the interest rate $r$, the drift value $\mu$ and the parameter $\alpha$.
	Moreover, the optimal capacity level increases in a linear rate with the innovation level $\theta$; it has a non-monotonic behaviour with the capacity of the old product $K_0$.
\end{itemize}




%Due to the independence of the contents presented along the thesis, the main conclusions were already pointed out on each chapter. Therefore, hereunder the most relevant achievements are summarized.

%Regarding both benchmark and capacity optimization models on chapters \ref{chapter:1} and \ref{chapter:2} we obtain similar results to the ones mentioned in the literature, particularly in \cite{dixit:book}: markets with larger volatility tend to delay the investment decision, while higher innovation levels tend to anticipate it.

On Chapter \ref{chapter:3} we conclude that the cannibalisation has a crucial role on the investment decision. For larger values of it, the firm is recommended to invest and immediately replace the \textit{old} by the \textit{new} product, being in the case treated on Chapter \ref{chapter:2}. On the other side, for smaller values of cannibalisation, a simultaneous production period followed by the total replacement of the \textit{old} product is more favourable that the immediate replacement. We conclude:
\begin{itemize}
	\item The demand value that triggers the introduction of the new product increases with the cannibalisation parameter $\eta$, the parameters $\alpha$ and $\delta$, the market's uncertainty $\sigma$ and the capacity of the old and new products $K_0$ and $K_1$; it decreases with the innovation level $\theta$;
	\item The demand value that triggers the removal of the old product increases with the market's uncertainty $\sigma$; it decreases with the cannibalisation parameter $\eta$, the parameter $\alpha$ and the capacity of the old and new products $K_0$ and $K_1$.
\end{itemize}
  
We extend the usual sensitivity analysis, that is narrowed to the comparative statics of the investment thresholds, by analysing the expected waiting times until the investment through numerical simulations. We observe that the investment waiting time follows an approximate symmetric distribution and decreases in a logarithmic rate with the initial demand value. Furthermore, the expected waiting time is observed to have a non-monotonic behaviour with the market's uncertainty, which indicates that volatility has a different impact on the waiting time and on the investment threshold.

Lastly, when analysing the impact of R\&D investment on the overall investment decision (by considering that we know the value of the demand at the instant the innovation breakthrough occurs), we conclude that it has no direct influence on the investment decision. Nevertheless, we are able to maximize the value function associate to the project by finding its optimal R\&D investment.

% ----------------------------------------------------------------------
\section{Suggestions for Future Work}
\label{section:future}

Although on this thesis many situations were explored from different perspectives, further studies can be carried out, by exploring the following aspects:
\begin{itemize}
	\item The demand evolves accordingly to a mean-revarting process (such as a Cox-Ingersoll-Ross or an Ornstein-Uhlenbeck process);
	
	\item The innovation process is modelled by a non-homogenous compound Poisson process or even a more general renewal process;
	
	\item Only the demand level at the time the R\&D investment is made is known and we (still) want to deduce the optimal amount to invest in R\&D as well as the best time to initiate the production of the innovative product, by exploring the situation referred on chapters \ref{chapter:1}, \ref{chapter:2} and \ref{chapter:3};
	
	\item We are able to estimate the parameters of proposed models with data provided from a real firm and, therefore, to help the firm in its investment decisions.
\end{itemize}

%Among many other suggestions, these are realist settings in the context of investment decisions upon innovative technological products relevant to nowadays IT firms' situation.

The above suggestions are feasible and natural continuation of this work and we believe that can contribute to improve the prediction accuracy of the optimal investment time, similarly to what was developed on this thesis.
