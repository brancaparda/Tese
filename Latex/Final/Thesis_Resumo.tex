%%%%%%%%%%%%%%%%%%%%%%%%%%%%%%%%%%%%%%%%%%%%%%%%%%%%%%%%%%%%%%%%%%%%%%%%
%                                                                      %
%     File: Thesis_Resumo.tex                                          %
%     Tex Master: Thesis.tex                                           %
%                                                                      %
%     Author: Andre C. Marta                                           %
%     Last modified :  2 Jul 2015                                      %
%                                                                      %
%%%%%%%%%%%%%%%%%%%%%%%%%%%%%%%%%%%%%%%%%%%%%%%%%%%%%%%%%%%%%%%%%%%%%%%%

\section*{Resumo}

% Add entry in the table of contents as section
\addcontentsline{toc}{section}{Resumo}

% Add entry in the table of contents as section
\addcontentsline{toc}{section}{Abstract}

Nesta tese é estudada a política de investimento óptima relativa a um produto tecnológico inovador, considerando os seguintes cenários:
\begin{enumerate}
	\item Uma empresa quer investir e entrar no mercado com um novo produto;
	\item Uma empresa já activa quer investir num novo produto, substituindo o antigo; 
	\item Uma empresa já activa quer investir num novo produto, permitindo um período de produção simultânea seguido da substituição total do produto antigo.
\end{enumerate}

Além disso, assumimos que a firma, quando activa, produz um produto estável no mercado e que a decisão de investimento é irreversível, instantanea e pode ser tomada em qualquer altura após o nível the inovação tecnológica desejado ser atingido, pagando os respectivos custos (não reembolsáveis).

Considerando que a procura evolui de acordo com um Movimento Geométrico Browniano e que o nível de inovação segue um Processo de Poisson Composto, derivamos o nível de procura que justifica o investmento para cada uma das situações referidas, seguido da sua análise comparativa. Estudamos ainda a sensibilidade do tempo até o investimento óptimo e o impacto do investimento em pesquisa e desenvolvimento (R\&D) no processo de investimento, através de análises quer analíticas quer numéricas.

Posto isto, esta tese pretende ser uma ajuda na actual indústria tecnológica, providenciando ferramentas poderosas para as equipas de decisão, assim como pelas suas contribuições para literatura em Matemática Financeira, em particular na área de Investimento sob Incerteza.






\vfill

\textbf{\Large Palavras-Chave:} Problemas de Paragem Óptima; Abordagem de Opções Reais; Investimento Sob Incerteza; Inovação Tecnológica.
