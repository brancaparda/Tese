%%%%%%%%%%%%%%%%%%%%%%%%%%%%%%%%%%%%%%%%%%%%%%%%%%%%%%%%%%%%%%%%%%%%%%%%
%                                                                      %
%     File: Thesis_Conclusions.tex                                     %
%     Tex Master: Thesis.tex                                           %
%                                                                      %
%     Author: Andre C. Marta                                           %
%     Last modified :  2 Jul 2015                                      %
%                                                                      %
%%%%%%%%%%%%%%%%%%%%%%%%%%%%%%%%%%%%%%%%%%%%%%%%%%%%%%%%%%%%%%%%%%%%%%%%

\chapter{Conclusions}
\label{chapter:conc}

Due to the independence of the contents presented along the thesis, the main conclusions were already pointed out on each chapter. Therefore, hereunder the most relevant achievements are summarized.

Regarding both benchmark and capacity optimization models on chapters \ref{chapter:1} and \ref{chapter:2} we obtain similar results to the ones mentioned in the literature, particularly in \cite{dixit:book}: markets with larger volatility tend to delay the investment decision, while higher innovation levels tend to anticipate it.

On Chapter \ref{chapter:3} we conclude that the cannibalisation has a crucial role on the investment decision. For larger values of it, the firm is recommended to invest and immediately replace the \textit{old} by the \textit{new} product. On the other side, for smaller values of cannibalisation, a simultaneous production period followed by the total replacement of the \textit{old} product is more favourable that the immediate replacement.
  
We extend the usual sensitivity analysis, that is narrowed to the comparative statics of the investment thresholds, by analysing the expected waiting times until the investment through numerical simulations. We observe that the investment waiting time follows an approximate symmetric distribution and decreases in a logarithmic rate with the initial demand value. Furthermore, the expected waiting time is observed to have a non-monotonic behaviour with the market's uncertainty, which indicates that volatility has a different impact on the waiting time and on the investment threshold.

Lastly, when analysing the impact of R\&D investment on the overall investment decision (by considering that we know the value of the demand at the instant the innovation breakthrough occurs), we conclude that it has no direct influence on the investment decision. Nevertheless, we are able to maximize the value function associate to the project by finding its optimal R\&D investment.

% ----------------------------------------------------------------------
\section{Future Work}
\label{section:future}

Although on this thesis many situations are explored from different perspectives, one could deepen the results obtained, by exploring the cases on which:
\begin{itemize}
	\item The demand evolves accordingly to a mean-revarting process (such as a Cox-Ingersoll-Ross or a Ornstein-Uhlenbeck process);
	
	\item The innovation process is modelled by a non-homogenous compound Poisson process or even a more general renewal process;
	
	\item Only the demand level at the time the R\&D investment is made is known and we (still) want to deduce the optimal amount to invest in R\&D as well as the best time to initiate the production of the innovative product, by exploring the situation referred on chapters \ref{chapter:1}, \ref{chapter:2} and \ref{chapter:3};
	
	\item We are able to estimate the parameters of proposed models with data provided from a real firm and, therefore, to help the firm in its investment decisions.
\end{itemize}

Among many other suggestions, these are realist settings in the context of investment decisions upon innovative technological products relevant to nowadays IT firms' situation.
