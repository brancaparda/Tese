%%%%%%%%%%%%%%%%%%%%%%%%%%%%%%%%%%%%%%%%%%%%%%%%%%%%%%%%%%%%%%%%%%%%%%%%
%                                                                      %
%     File: Thesis_Abstract.tex                                        %
%     Tex Master: Thesis.tex                                           %
%                                                                      %
%     Author: Andre C. Marta                                           %
%     Last modified :  2 Jul 2015                                      %
%                                                                      %
%%%%%%%%%%%%%%%%%%%%%%%%%%%%%%%%%%%%%%%%%%%%%%%%%%%%%%%%%%%%%%%%%%%%%%%%

\section*{Abstract}

% Add entry in the table of contents as section
\addcontentsline{toc}{section}{Abstract}

In this thesis the optimal investment policy regarding an innovative technological product is addressed under the following scenarios:
\begin{enumerate}
	\item A firm wants to invest and enter the market with a new product;
	\item An active firm wants to invest and launch a new product, that totally replaces the old one; 
	\item An active firm wants to invest and launch a new product, allowing a temporarily simultaneous production period, followed by the old product's total replacement.
\end{enumerate}

Moreover, we assume that the firm, when active, produces an established product and that the investment decision is irreversible, instantaneous, has an associate (sunk) cost and can be made at any time after a desired innovation level is reached.

Considering the demand to evolve as a Geometric Brownian Motion and innovation level accordingly to a Compound Poisson Process, we derive the demand level(s) that justifies the investment decision(s) in each situation along with the respective comparative statics analysis. Lastly, the sensitivity of optimal investment times and the impact of R\&D investment in the innovation process are also analysed either analytically or numerically. 

Overall, this thesis expects to complement modern IT industry, providing powerful tools for decision support teams, and to bring relevant contributions to the literature on Financial Mathematics, particularly to the Investment under Uncertainty field.





\vfill

\textbf{\Large Keywords:} Optimal Stopping Problems; Real Options Approach; Investment Under Uncertainty; Technology Innovation.

