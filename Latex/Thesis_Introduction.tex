%%%%%%%%%%%%%%%%%%%%%%%%%%%%%%%%%%%%%%%%%%%%%%%%%%%%%%%%%%%%%%%%%%%%%%%%
%                                                                      %
%     File: Thesis_Introduction.tex                                    %
%     Tex Master: Thesis.tex                                           %
%                                                                      %
%     Author: Andre C. Marta                                           %
%     Last modified :  2 Jul 2015                                      %
%                                                                      %
%%%%%%%%%%%%%%%%%%%%%%%%%%%%%%%%%%%%%%%%%%%%%%%%%%%%%%%%%%%%%%%%%%%%%%%%

\chapter{Introduction}
\label{chapter:introduction}

Insert your chapter material here...

%%%%%%%%%%%%%%%%%%%%%%%%%%%%%%%%%%%%%%%%%%%%%%%%%%%%%%%%%%%%%%%%%%%%%%%%
\section{Motivation}
\label{section:motivation}

Relevance of the subject...
Example
Goals in the end of this mini chapter.

%%%%%%%%%%%%%%%%%%%%%%%%%%%%%%%%%%%%%%%%%%%%%%%%%%%%%%%%%%%%%%%%%%%%%%%%
\section{Optimal Stopping Problems}
\label{section:osp}

In this thesis we use the Real Options framework to study three different investment decisions in which the demand is considered to be a diffusion process.
%%%%%%%%%%%%%%%%%%%%%%%%%%%%%%%%%%%%%%%%%%%%%%%%%%%%%%%%%%%%%%%%%%%%%%%%

\subsection{Literature}

Quite many recent works
%(Farzin \textit{et al.} (1998) \cite{farzin:cap}, for instance) 
suggest that Real Option\footnote{Real Option's concept will be defined on Section \ref{bc_ro}.} analysis is much more advantageous than the Net Present Value (NPV) analysis when it comes to support investment decisions, since the last one assumes that the company has a now or never approach regarding the investment decision, ignoring the freedom in timing.
%and hence it ignores the option of having a better situation in the future. 
On the other side a Real Option Analysis takes into account the future potential, as well the respective uncertainty.
%Among many studies already made, Farzin \textit{et al.} (1998) \cite{farzin:cap}, do a comparative study between both approaches in the technology adoption context.

The first contributions on the Real Options approach were due to McDonald \& Siegel (1986)  \cite{siegel}, in whose work they model an investment problem where the investor must decide when it is the best time to exercise, taking into account that the value of the investment project is stochastically random, evolving accordingly to a Geometric Brownian Motion; and due to Dixit (1989) \cite{dixit_alone}, in whose work he models the best time to make entry and exit decisions, while considering that there might be costs associated to each decision and that the market price evolves accordingly to a Brownian Motion.

Years later, Dixit \& Pindyck (1994) \cite{dixit:book} publish what is considered by some of the experts as the (financial) \textit{Bible of Real Options approach}. It exploits an analogy between real options and financial investment decisions, focusing on many different decision problems (entry, investment and exit, among them) dependent on different stochastically behavioured (diffusion processes and jump diffusion processes, among them) measures, such as demand or market price.

Some other topics regarding Real Option increased their relevance. More particularly, the optimal production capacity to be chosen and the impact of technology adoption - being this last one a field of increasing interest nowadays.

Regarding optimal production capacity, Huisman \& Kort (2013) \cite{huis:cap} consider a competitive market in which a monopoly and a duopoly are inserted and want to deduce the best time and capaciy to invest in a new product. 

Regarding technology adoption,
Farzin \textit{et al.} (1998) \cite{farzin:cap}, do a comparative study between NPV and Real Option approaches un the context that a company wants to deduce when is the best time and level to invest in a technology.
More recently, Hagspiel \textit{et al.} (2016) \cite{hagspiel:cap} model the best time for a firm to invest in a new product, having always in mind the option to exit the market, while considering that the firm faces a declining profit for the established product and that the demand level evolves accordingly to a GBM. %The solution leads to three different demand thresholds for the respective possible decisions.

In this same year, Pimentel (2018) \cite{rita} explored both. By considering two sources of uncertainty (the demand which evolves as a jump diffusion process and the innovation process which evolves as a compound Poisson process), she models the optimal times for a firm, which is producing an established product, to invest in a new product and to stop the production of the established product. Regarding the described situation, two models were developed: the benchmark model and the capacity optimization model.



%In this work we will also consider the demand to be stochastically, evolving accordingly to a Geometric Brownian Motion. On the third optimal stopping problem, on Section \ref{chapter:3}, we will also derive three different thresholds, although these are related with the possibility of either invest in a new product with simultaneous production of the new and the established product and then stop the production of the established product or invest on the new product and immediately replace the established product. A similar situation was already presented by Pimentel \cite{rita}, however in a different context. On her work, she considers two sources of uncertain - which strongly influence each of the thresholds derived -, while in our work we only consider one, the demand level already referred. 




%%%%%%%%%%%%%%%%%%%%%%%%%%%%%%%%%%%%%%%%%%%%%%%%%%%%%%%%%%%%%%%%%%%%%%%%
\section{Thesis Outline}
\label{section:outline}

In this thesis we will treat different situations related with the irreversible decision of investing in a new product under one source of uncertainty.

On Chapter \ref{chapter:bc}, the main concepts and state-of-the-art (common to all problems) will be introduced. More specifically:
\begin{itemize}
	\item On Section \ref{section:scop} we introduce the class of stochastic control problems and then specify in the class optimal stopping problems, following a similar approach as presented in \cite{ross}.
	
	\item On Section \ref{section:osro}, we show how optimal stopping problems relate to investment decisions under uncertainty,  following a similar approach as presented in \cite{dixit:book}.
\end{itemize}

On Chapter \ref{ }, the first model is derived. Here we consider the situation in which a firm wants to find the optimal time to introduce a new product, having none being produced at the moment. More specifically:
\begin{itemize}
	\item On Section \ref{label}, we derive the optimal decision regarding the original cash-flow.
	
	\item On Section \ref{}, we derive the optimal decision regarding the maximized cash-flow with respect to production capacity.
	
	\item On Section \ref{label}, we study the behaviour of the decision threshold with the different parameters.
\end{itemize}

On Chapter \ref{label}, we consider the situation in which a firm has a \textit{stable} and recognized product in the market and wants to find the optimal time to invest in a new product while, at the same instant, replace the \textit{stable} product by the new one.  More specifically:
\begin{itemize}
	\item On Section \ref{label}, we derive the optimal decision regarding the original cash-flow.
	
	\item On Section \ref{}, we derive the optimal decision regarding the maximized cash-flow with respect to production capacity of the new product.
	
	\item On Section \ref{label}, we study the behaviour of the decision threshold with the different parameters.
\end{itemize}

On Chapter \ref{label}, we consider a similar situation as in Chapter \ref{}, but with the extra possibility of producing both products simultaneously. Therefore, now the firm wants to find the optimal time to invest in a new product and the optimal time to stop producing the \textit{stable} product.  More specifically:
\begin{itemize}
	\item On Section \ref{label}, we derive the optimal decisions regarding the original cash-flows.
\end{itemize}

On Chapter \ref{label}, we derive the optimal R\&D investment, by maximizing the expected value function function with respect to the innovation process. More specifically:
\begin{itemize}
	\item On Section \ref{label}, we derive the optimal R\&D investment considering that the innovation process only takes one jump to achieve the breakthrough level.
	
	\item On Section \ref{}, we generalize the previous section, by considering that the innovation process takes $n \in \mathds{N}$ jumps to achieve the breakthrough level.
	
	\item On Section \ref{label}, we study the behaviour of the decision threshold with the different parameters.
\end{itemize}


On Chapter \ref{label}, we perform some simulations.

Finally, on Chapter \ref{label}, we summarize the relevant findings of the work done and how it can be extended.




\subsection{Some Notation}
\label{intro:notation}

Throughout the chapters, many terms will appear and their explanation will come along. However most of them will be always the same, since they do not depend on the chapter that we are working on. Therefore, to promote a better understanding in the context of the problem, the major notation (and how they are restricted on problem domain) will be now introduced:
\begin{itemize}
	\item $\{ W(t), \ t \geq 0 \}$: Standard Brownian Motion (or Wiener Process) which is a stochastic process that has the following characteristics:
	\begin{enumerate}
		\item $W(0) = 0$ with probability 1;
		\item $W(t) - W(s) \sim N(0, t-s)$. Notice that $\mathds{E}[W(t)] = 0$ and $Var[W(t)] = t$;
		\item Independent increments: $\forall \ 0 < s_i < t_i < s_j < t_j: \ W(t_i) - W(s_i) \amalg W(t_j) - W(s_j) $;
		
		Stationary increments:  $\forall t,s \geq 0: \ W(t+s) - W(s) \buildrel d\over= W(t) $;
		\item $W(t)$ is continuous in $t$ (however nowhere differentiable).
	\end{enumerate}
It is also seen as the continuous version of a Random Walk with Normal increments.

	\item  $\{ X(t), \ t \geq 0 \}$: Geometric Brownian Motion (GBM) represents the demand for a certain product at each instant $t$. It is the solution of the following stochastic differential equation (SDE)
	$$ dX_t=\mu X_t dt + \sigma X_t d W_t, \ X_0=x, $$
	where $\mu$ represents the drift and $\sigma$ the volatility of the demand.
	
	\item $R$: R\&D costs such as size of laboratories, wages of the scientists, their computers/machines, \textit{etecetera}, directly relate with the innovation process. These are seen as sunk costs, that is, costs that cannot be recovered after being incurred.
	
	\item  $\{ \theta(t), \ t \geq 0 \}$: innovation process assumed to be a homogeneous Compound Poisson Process, that is a stochastic process that evolves accordingly to
	$$\theta_t= \theta_0+ u N_t$$
	where $\theta_0$ corresponds to the initial innovation level, $u > 0$ is the jump size and $\{N_t, \ t \geq 0\}$ follows a Poisson process with rate $\lambda(R)=R^\gamma, \ \gamma \in (0,1)$. This rate function is such that $\lambda(0) = 0$: no R\&D means zero probability of innovating; $\lambda^\prime (R)>0$: bigger investment means the higher probability of success and $ \lambda ^{ \prime \prime} (R)<0$: exists a amount of R\&D costs that maximizes the rate function, that is, $\exists R^*: \lambda(R^*)\geq \lambda(R) \  \forall R$.
	
	\item $\theta$: innovation breakthrough level. That is, the level of innovation for which we decide to invest in the new product. Considered to be reached in $n \in \mathds{N}$ jumps, as it will be seen on Chapter \ref{chapter:max}.
	
	\item $K_i$: capacity of production of product $i$. Note that when a single product is considered, there is no mention to index $i$. The firm is considered to produce always up to its capacity, allowing us to consider $K_i$ as the quantity produced. Since profit functions need to be positive, on Chapters \ref{chapter:2} and \ref{chapter:3}, we have the following restrictions regarding capacities of \textit{old} and \textit{new} product, respectively, $K_0<1/\alpha$ and $K_1<\theta/\alpha$. Note that (only) the last restriction will also hold for Chapter \ref{chapter:1}.
	
	\item $\alpha$: constant parameter that reflects the sensitivity of the quantity with respect to the price, $\alpha>0$.
	
	\item $\delta$: constant parameter that reflects the sensitivity of the quantity with respect to investment sunk costs. These sunk costs will be denoted by $\delta K_1, \  \delta>0$ (or $\delta K$, on Chapter \ref{chapter:1}).
	
	\item $\eta$: cannibalization parameter corresponding to the crossed effect between the old and the new product. It's seen as a penalty representing how the quantity associated to a product will influence the price of the other. On Chapter \ref{chapter:3}, we consider that this influence is the same for both products, resulting in a unique cannibalisation parameter. This one cannot be greater than the sensibility parameter $\alpha$, that is, $\eta <\alpha$.
\end{itemize}