%%%%%%%%%%%%%%%%%%%%%%%%%%%%%%%%%%%%%%%%%%%%%%%%%%%%%%%%%%%%%%%%%%%%%%%%
%                                                                      %
%     File: Thesis_Resumo.tex                                          %
%     Tex Master: Thesis.tex                                           %
%                                                                      %
%     Author: Andre C. Marta                                           %
%     Last modified :  2 Jul 2015                                      %
%                                                                      %
%%%%%%%%%%%%%%%%%%%%%%%%%%%%%%%%%%%%%%%%%%%%%%%%%%%%%%%%%%%%%%%%%%%%%%%%

\section*{Resumo}

% Add entry in the table of contents as section
\addcontentsline{toc}{section}{Resumo}

% Add entry in the table of contents as section

O tema desta tese insere-se na área de Matemática Financeira, em particular em Investimento sob Incerteza. O seu objectivo é definir a política de investimento óptima relativa a um produto tecnológico inovador, através da maximização do seu lucro esperado,
%. Esta tese estuda estratégias de apoio à decisão em produtos tecnológicos inovadores,
%Nesta tese é estudada a política de investimento óptima relativa a um produto tecnológico inovador,
relativa aos seguintes cenários:
\begin{enumerate}
	\item Uma empresa quer investir e entrar no mercado com um produto novo;
	\item Uma empresa já activa quer investir num novo produto, substituindo o antigo; 
	\item Uma empresa já activa quer investir num novo produto, permitindo um período de produção simultânea seguido da substituição total do produto antigo.
\end{enumerate}

Assume-se também que a empresa, quando activa, produz um produto estável no mercado e que a decisão de investimento é irreversível, instantânea e pode ser tomada em qualquer altura após o nível de inovação tecnológica desejado ser atingido e cujos custos não são reembolsáveis.

A metodologia de estudo assume que a procura no mercado
%Considerando que a procura 
evolui de acordo com um Movimento Geométrico Browniano e que o nível de inovação segue um Processo de Poisson Composto. Com base nestas condições, deriva-se o nível de procura que justifica o investimento para cada uma das situações referidas, seguindo-se da sua análise comparativa. Estuda-se também a sensibilidade do tempo até o investimento óptimo e o impacto do investimento em investigação e desenvolvimento (R\&D) no processo de investimento, analiticamente e numericamente.

Espera-se que esta tese possa auxiliar as equipas de decisão a avaliar os investimentos relacionados com produtos tecnológicos, indicando a melhor altura para se investir, qual o valor de capacidade de produção óptimo e o valor do projecto.
%Posto isto, esta tese pretende ser uma ajuda na actual indústria tecnológica, providenciando ferramentas poderosas para as equipas de decisão, assim como pelas suas contribuições para literatura em Matemática Financeira, em particular na área de Investimento sob Incerteza.

%\colorbox{red}{Referir objectivo na tese. Adaptar último parágrafo, ver crítica do tio}




\vfill

\textbf{\Large Palavras-Chave:} Problemas de Paragem Óptima; Abordagem de Opções Reais; Investimento Sob Incerteza; Inovação Tecnológica.
