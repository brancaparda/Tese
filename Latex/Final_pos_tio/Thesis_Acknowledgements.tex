%%%%%%%%%%%%%%%%%%%%%%%%%%%%%%%%%%%%%%%%%%%%%%%%%%%%%%%%%%%%%%%%%%%%%%%%
%                                                                      %
%     File: Thesis_Acknowledgments.tex                                 %
%     Tex Master: Thesis.tex                                           %
%                                                                      %
%     Author: Andre C. Marta                                           %
%     Last modified :  2 Jul 2015                                      %
%                                                                      %
%%%%%%%%%%%%%%%%%%%%%%%%%%%%%%%%%%%%%%%%%%%%%%%%%%%%%%%%%%%%%%%%%%%%%%%%

\section*{\acknowledgments}

% Add entry in the table of contents as section
\addcontentsline{toc}{section}{\acknowledgments}

Apesar deste trabalho estar escrito em inglês não resisto em deixar a nota mais pessoal na minha língua materna.

Começo por agradecer à minha orientadora, a Professora Cláudia,  que muito em cima da hora e a uma distância de quase 2000 km decidiu guiar-me no culminar de cinco anos de estudo. 
Os desafios rapidamente surgiram, assim como as resultantes dúvidas, os progressos na direcção errada e os muito celebrados resultados finais. 
Foi maravilhosa a oportunidade de me ter dado a conhecer as pessoas com quem de perto trabalha e com elas ter partilhado os meus progressos; estupendamente valioso o seu voto de confiança para trabalhar onde me aprouvesse (permitindo-me obter alguns progressos mais relevantes fora de território nacional); e incansável a sua paciência no ping-pong final deste trabalho.

E porque a tese não é fruto apenas dos seis meses em que nela se trabalha, mas antes sim de todo o percurso académico, fora de casa passado, não deixo de agradecer ao pessoal de LMAC que quase família se tornou, pelos muitos cafés, sessões de estudo, churrascos e devaneios; às \textit{soirées do Bob} que sempre serviram para desanuviar das tempestades matemáticas; a todos os amigos que Lausanne me deu e que me ensinaram que as saudades não nos dominam se estivermos em boa companhia e que se consegue realizar (com sucesso!) enormes quantidades de trabalho conjugando muitas horas pelo Rolex e pelo Departamento de Matemática com passeios nos Alpes, idas ao Léman, frutos secos e lutas com colheres de pau no final do dia; e a todos aqueles que surgiram e ficaram (ou se foram), deixando a sua pegada impressa naquilo que hoje sou.

Também, como não poderia deixar de ser, um enorme obrigada a toda a minha família por todo o apoio e conselhos que me dão, por assegurarem que o meu calmo e confortável ninho se encontra sempre de braços abertos para me receber e por todas as infidáveis refeições em que nos reunimos que sempre me deixam de coração cheio. 

Quanto ao futuro, esse é um mistério! Contudo, e por enquanto, toda a minha juventude exclama as palavras de Jack London em "The Call of the Wild",
\begin{flushright}
	\textit{He was mastered by the sheer surging of life,\\
		the tidal wave of being,\\ 
		the perfect joy of each separate muscle, joint,\\
		and sinew in that it was everything that was not death,\\ that it was aglow and rampant,\\
		expressing itself in movement,\\
		flying exultantly under the stars.}
\end{flushright}



