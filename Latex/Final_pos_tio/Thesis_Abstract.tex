%%%%%%%%%%%%%%%%%%%%%%%%%%%%%%%%%%%%%%%%%%%%%%%%%%%%%%%%%%%%%%%%%%%%%%%%
%                                                                      %
%     File: Thesis_Abstract.tex                                        %
%     Tex Master: Thesis.tex                                           %
%                                                                      %
%     Author: Andre C. Marta                                           %
%     Last modified :  2 Jul 2015                                      %
%                                                                      %
%%%%%%%%%%%%%%%%%%%%%%%%%%%%%%%%%%%%%%%%%%%%%%%%%%%%%%%%%%%%%%%%%%%%%%%%

\section*{Abstract}

% Add entry in the table of contents as section
\addcontentsline{toc}{section}{Abstract}

The scope of this thesis is Financial Mathematics, in particular Investment under Uncertainty.
Its goal is to define the optimal investment policy regarding an innovative technological product, by maximizing its expected long run profit, related to
%This thesis studies mathematical decision tools that can guide investors in novel technological products by taking into account
%support decisions strategies 
%In this thesis the optimal investment policy regarding an innovative technological product is addressed under 
the following scenarios:
\begin{enumerate}
	\item A firm wants to invest and enter the market with a new product;
	\item An active firm wants to invest and launch a new product, that totally replaces the old one; 
	\item An active firm wants to invest and launch a new product, while keeping temporarily the old product.
	%allowing a temporarily simultaneous production period, followed by the old product's total replacement.
\end{enumerate}

Moreover,
% we assume that the firm,
it is assumed  that an active firm produces an established product. The investment decision is irreversible, instantaneous, has an associate (sunk) cost and can be made at any time after a desired innovation level is reached.

%Considering
The methodology of this thesis assumes the demand to evolve as a Geometric Brownian Motion and the innovation level accordingly to a Compound Poisson Process and derives the demand level(s) that justifies the investment decision(s) in each situation along with the respective comparative statics analysis. The sensitivity of optimal investment times and the impact of R\&D investment in the innovation process are also analysed both analytically and numerically.

Overall, this thesis expects to support decision teams with technological products' investments, stating when is the best time to invest, the optimal production capacity and the value of the project.
% complement modern IT industry, providing powerful tools for decision support teams, and to bring relevant contributions to the literature on Financial Mathematics, particularly to the Investment under Uncertainty field.

%\colorbox{red}{Referir objectivo na tese. Adaptar último parágrafo, ver crítica do tio}



\vfill

\textbf{\Large Keywords:} Optimal Stopping Problems; Real Options Approach; Investment Under Uncertainty; Technology Innovation.

