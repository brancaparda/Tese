%%%%%%%%%%%%%%%%%%%%%%%%%%%%%%%%%%%%%%%%%%%%%%%%%%%%%%%%%%%%%%%%%%%%%%%%
%                                                                      %
%     File: Thesis_Appendix_B.tex                                      %
%     Tex Master: Thesis.tex                                           %
%                                                                      %
%     Author: Andre C. Marta                                           %
%     Last modified :  2 Jul 2015                                      %
%                                                                      %
%%%%%%%%%%%%%%%%%%%%%%%%%%%%%%%%%%%%%%%%%%%%%%%%%%%%%%%%%%%%%%%%%%%%%%%%

\chapter{\texttt{Mathematica} code to assess the influence of the number of innovation jumps}
\label{chapter:appendixjump}

In this appendix we present the code used to make the assessments presented on Chapter \ref{chapter:max}.
\vspace{0.5cm}
% ----------------------------------------------------------------------

$\bullet$ Calculation of the optimal R\&D investment, $R^*$, for given parameter $\gamma$ and number of jumps $n$:
\begin{lstlisting}
V[R_] := (R^gamma/(r + R^gamma))^n *F - R;
ptstat = Flatten[Values[NSolve[r R + R^(1 + gamma) - 
	F n r (R^gamma/(r + R^gamma))^n gamma == 0 && R > 0,R]]]; (*raizes
		 da 1a derivada *)
If[ptstat == {},
	ptstat = 
	Flatten[Values[FindRoot[r R + R^(1 + gamma) -
		F n r (R^gamma/(r + R^gamma))^n gamma == 0, {R, 
		0.5}]]]  (* outro metodo para calcular raizes da 1a derivada *)
];
d2[R_] := (F n r (R^gamma/(r + R^gamma))^n gamma (-R^gamma (1 + gamma) 
	+ r (-1 + n gamma)))/(R^2 (r + R^gamma)^2);
maxrelat = Negative@Map[d2, ptstat] //. False -> 0 //. True -> 1;  (* 
	pontos estacionarios com 2a derivada <0 *)
neg = maxrelat*ptstat; (* vector com entradas dadas por pontos est. 
	com 2a derivada <0, cc entrada a zero *)
If[Norm[maxrelat] == 1, (* existe apenas 1 ponto est. com 2a derv. <
0 -> maximo global! *)
	max = Norm[ptstat*neg/Norm[neg]],
	max = 0;
	tam = Length[ptstat];
	While[i <= tam, (* teste para escolher qual dos pontos est. 
com 2a derv. <0 -> maximo global *)
		If[V[maxrelat[[i]]] > max,
		max = maxrelat[[i]]];
		i++];
	];
max];
\end{lstlisting}


\vspace{3mm}
$\bullet$ Calculation of the optimal R\&D investment, $R^*$, up to 5 innovation jumps and by varying $\gamma$ from 0 to 1:

\begin{lstlisting}
n = 1;
RoptSaltos = {};
While[n <= 5,
	gammas = Rest[Range[0, 1, 0.05]];
	Ropt = MapThread[calcR, {gammas, Table[n, {i, 1, Length[gammas]}]}];
		(* lista R optimais *)
	RoptSaltos = Append[RoptSaltos, Ropt];
	n++
];
\end{lstlisting}