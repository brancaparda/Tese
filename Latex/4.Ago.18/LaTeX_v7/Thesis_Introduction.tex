%%%%%%%%%%%%%%%%%%%%%%%%%%%%%%%%%%%%%%%%%%%%%%%%%%%%%%%%%%%%%%%%%%%%%%%%
%                                                                      %
%     File: Thesis_Introduction.tex                                    %
%     Tex Master: Thesis.tex                                           %
%                                                                      %
%     Author: Andre C. Marta                                           %
%     Last modified :  2 Jul 2015                                      %
%                                                                      %
%%%%%%%%%%%%%%%%%%%%%%%%%%%%%%%%%%%%%%%%%%%%%%%%%%%%%%%%%%%%%%%%%%%%%%%%

\chapter{Introduction}
\label{chapter:introduction}

Insert your chapter material here...

%%%%%%%%%%%%%%%%%%%%%%%%%%%%%%%%%%%%%%%%%%%%%%%%%%%%%%%%%%%%%%%%%%%%%%%%
\section{Motivation}
\label{section:motivation}

Relevance of the subject...
Example
Goals in the end of this mini chapter.

%%%%%%%%%%%%%%%%%%%%%%%%%%%%%%%%%%%%%%%%%%%%%%%%%%%%%%%%%%%%%%%%%%%%%%%%
\section{Optimal Stopping Problems}
\label{section:osp}

Definição.

Princípio de Programação Dinâmica.

HJB + Gerador infinitesimal de GBM

Investimento: região de continuação e de paragem

Outro tipo de problemas (para além dos de investimento)

Teorema da Verificação

%%%%%%%%%%%%%%%%%%%%%%%%%%%%%%%%%%%%%%%%%%%%%%%%%%%%%%%%%%%%%%%%%%%%%%%%
%\section{Objectives}
%\label{section:objectives}

%xplicitly state the objectives set to be achieved with this thesis...
%Problema de tempo de paragem e problema de maximização refindo os três contextos

%%%%%%%%%%%%%%%%%%%%%%%%%%%%%%%%%%%%%%%%%%%%%%%%%%%%%%%%%%%%%%%%%%%%%%%%
\section{Thesis Outline}
\label{section:outline}

In this thesis we will treat different situations related with the decision of investing in a new product under uncertainty.

On Chapter \ref{ }, the main concepts and state-of-the-art (common to all problems) will be introduced. More specifically:
\begin{itemize}
	\item On Section \ref{label} we introduce the class of optimal stopping problems, following a similar approach as presented in \cite{ross}.
	
	\item On Section \ref{}, we show how optimal stopping problems relate to investment decisions under uncertainty,  following a similar approach as presented in \cite{dixit:book}.
\end{itemize}

On Chapter \ref{ }, the first model is derived. Here we consider the situation in which a firm wants to find the optimal time to introduce a new product, having none being produced at the moment. More specifically:
\begin{itemize}
	\item On Section \ref{label}, we derive the optimal decision regarding the original cash-flow.
	
	\item On Section \ref{}, we derive the optimal decision regarding the maximized cash-flow with respect to production capacity.
	
	\item On Section \ref{label}, we study the behaviour of the decision threshold with the different parameters.
\end{itemize}

On Chapter \ref{label}, we consider the situation in which a firm has a \textit{stable} and recognized product in the market and wants to find the optimal time to invest in a new product and, at the same instant, replace the \textit{stable} product by the new one.  More specifically:
\begin{itemize}
	\item On Section \ref{label}, we derive the optimal decision regarding the original cash-flow.
	
	\item On Section \ref{}, we derive the optimal decision regarding the maximized cash-flow with respect to production capacity of the new product.
	
	\item On Section \ref{label}, we study the behaviour of the decision threshold with the different parameters.
\end{itemize}

On Chapter \ref{label}, we consider a similar situation as in Chapter \ref{}, but with the extra possibility of producing both products simultaneously. Therefore, now the firm wants to find the optimal time to invest in a new product and the optimal time to stop producing the \textit{stable} product.  More specifically:
\begin{itemize}
	\item On Section \ref{label}, we derive the optimal decisions regarding the original cash-flows.
\end{itemize}

On Chapter \ref{label}, we derive the optimal R\&D investment, by maximizing the expected value function function with respect to the innovation process. More specifically:
\begin{itemize}
	\item On Section \ref{label}, we derive the optimal R\&D investment considering that the innovation process only takes one jump to achieve the breakthrough level.
	
	\item On Section \ref{}, we generalize the previous section, by considering that the innovation process takes $n \in \mathds{N}$ jumps to achieve the breakthrough level.
	
	\item On Section \ref{label}, we study the behaviour of the decision threshold with the different parameters.
\end{itemize}


On Chapter \ref{label}, we perform some simulations.

Finally, on Chapter \ref{label}, we summarize the relevant findings of the work done and how it can be extended.




\subsection{Some Notation}

Throughout the chapters, many terms will appear and their explanation will come along. However most of them will be always the same, since they do not depend on the chapter that we are working one. Therefore, to promote a better understanding in the context of the problem, the major notation (and its restriction) will be now introduced:
\begin{itemize}
	\item $\{ W(t), t \geq 0 \}$: Standard Brownian Motion (or Wiener Process) which is a stochastic process that has the following characteristics:
	\begin{enumerate}
		\item $W(0) = 0$ with probability 1;
		\item $W(t) - W(s) \sim N(0, t-s)$. Notice that $\mathds{E}[W(t)] = 0$ and $Var[W(t)] = t$;
		\item Independent increments: $\forall 0 < s_i < t_i < s_j < t_j: \ W(t_i) - W(s_i) \amalg W(t_j) - W(s_j) $;
		
		Stationary increments:  $\forall t,s \geq 0: \ W(t+s) - W(s) \buildrel d\over= W(t) $;
		\item $W(t)$ is continuous in $t$ (however is nowhere differentiable).
	\end{enumerate}
It is also seen as the continuous version of a Random Walk with Normal increments.

	\item  $\{ X(t), t \geq 0 \}$: Geometric Brownian Motion (GBM) that represents the demand for a certain product. It is the solution of the following stochastic differential equation (SDE)
	$$ dX_t=\mu X_t dt + \sigma X_t d W_t, \ X_0=x, $$
	where $\mu$ represents the drift and $\sigma$ the volatility, both associated to the demand.
	
	\item $R$: R\&D costs such as size of laboratories, wages of the scientists, their computers/machines, \textit{etc}.
	
	\item  $\{ \theta(t), t \geq 0 \}$: innovation process assumed to be a homogeneous Compound Poisson Process, that is 
	$$\theta_t= \theta_0+ u N_t$$
	where $\theta_0$ corresponds to the initial innovation level, $u > 0$ is the jump size and $\{N_t, t \geq 0\}$ follows a Poisson process with rate $\lambda(R)=R^\gamma, \ \gamma \in (0,1)$. This rate function is such that $\lambda(0) = 0$ (no R\&D means zero probability of innovating); $\lambda^\prime (R)>0$ (bigger investment means the higher probability of success) and $ \lambda ^{ \prime \prime} (R)>0$.
	
	\item $\theta$: innovation breakthrough level. That is, the level of innovation for which we decide to invest in the new product.
	
	\item $K_i$: capacity of production of product $i$. Note that when a single product is considered, there is no mention to index $i$. The firm is considered to produce always up to its capacity. Since profit functions need to be positive, we have the following restrictions regarding capacities of \textit{old} and \textit{new} product, respectively, $K_0<1/\alpha$ and $K_1<\theta/\alpha$.
	
	\item $\alpha$: constant parameter that reflects the sensitivity of the quantity with respect to the price.
	
	\item $\delta$: constant parameter that reflects the sensitivity of the quantity with respect to investment sunk costs, that is, costs that cannot be recovered after investing. These sunk costs will be denoted by $\delta K_1$ (or $\delta K$, on Chapter \ref{chapter:2}).
	
	\item $\eta$: cannibalization parameter corresponding to the crossed effect between the old and the new product. It's seen as a penalty representing how the quantity associated to a product will influence the price of the other. On Chapter \ref{chapter:4}, we consider that this influence is the same for both products, so we there is a unique cannibalisation parameter. This cannot be greater than the sensibility parameter $\eta <\alpha$.
\end{itemize}