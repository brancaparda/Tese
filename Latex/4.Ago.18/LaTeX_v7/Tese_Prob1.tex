%%%%%%%%%%%%%%%%%%%%%%%%%%%%%%%%%%%%%%%%%%%%%%%%%%%%%%%%%%%%%%%%%%%%%%%%
%                                                                      %
%     File: Thesis_Background.tex                                      %
%     Tex Master: Thesis.tex                                           %
%                                                                      %
%     Author: Andre C. Marta                                           %
%     Last modified :  2 Jul 2015                                      %
%                                                                      %
%%%%%%%%%%%%%%%%%%%%%%%%%%%%%%%%%%%%%%%%%%%%%%%%%%%%%%%%%%%%%%%%%%%%%%%%

\chapter{Investing and entering the market with a new product (Firm is not active before investing)}
\label{chapter:background}

Insert your chapter material here...


%%%%%%%%%%%%%%%%%%%%%%%%%%%%%%%%%%%%%%%%%%%%%%%%%%%%%%%%%%%%%%%%%%%%%%%%
\section{Introduction}
\label{section:overview}

In this chapter we consider a firm that wants to invest in a product, after a certain innovation level $\theta$ is reached, and to produce it in long term. To do so, the firm needs to incur an investment cost proportional to the capacity production $K_1$. This cost is given by $\delta K_1$, with $\delta>0$ a sensibility parameter related to the investment. We consider here that, at the investing time, the firm needs to pay the investment cost and that the production starts immediately.


The demand function associated to the product to be introduced evolves stochastically with the demand process \textbf{X} and it is given by
\begin{equation}
p(X_t)=(\theta-\alpha K) X_t \geq 0
\label{prob1:pi}
\end{equation}
where $\alpha>0$ is a sensibility parameter and $X_t$ corresponds to the demand level observed at the instant $t\geq0$.

Multiplying the demand function by production capacity we obtain the associated instantaneous profit $\pi$, that is given by
\begin{equation}
\pi(X_t)=(\theta-\alpha K)K X_t \geq 0.
\label{prob1:pi}
\end{equation}


The firm is considered to produce always up to its capacity and that the variable costs are constant. Also, time is set to start when the innovation process reaches $\theta$, since before it it's useless to make an investment decision. Therefore $X_0$ refers to the demand level observed when the desired innovation level is reached. These facts will simplify our notation without losing the applicability of the model.

As mentioned before, two models will be derived. The first one corresponds to the benchmark model. The simplest model to be considered. The second one will take into account the maximized instantaneous profit in function of the production capacity $K$.


%%%%%%%%%%%%%%%%%%%%%%%%%%%%%%%%%%%%%%%%%%%%%%%%%%%%%%%%%%%%%%%%%%%%%%%%
\section{Stopping Problem}
\label{section:1_theory}



\subsection{Benchmark Model}
\label{subsec:1_bm}

We start with the simplest model. In the benchmark model we want to find when is the optimal time invest in the product (in the sense that maximizes the expected long-term profit), taking into account all the information previously referred.

Denoting the time of investment in the new product as $\tau$, our optimization problem can be written as 

In the benchmark model we want to find when is the optimal time invest in the product (in the sense that maximizes the expected long-term profit), taking into account all the information previously referred.

Denoting the time of investment in the new product as $\tau$, our optimization problem can be written as 
\begin{equation}
\sup_\tau \textbf{E}^{X_0=x} \left[e^{-r\tau }\left( \int_\tau^\infty e^{-r(\tau-s)} \pi(X_s)\ ds -\delta K \right) \ \textbf{1}_{\{\tau<\infty\}} \right]
\label{eq:probjj}
\end{equation}
for $\theta, x\in R^+$. Here $\textbf{E}^{X_0=x}\left[ \ . \ \right]$ corresponds to the expected value conditional to $X_0=x$, that is, $\textbf{E} \left[ \ . \ | \ X_0=x \right]$.

We can simplify this expression. Using Tower rule and conditioning on the instant when we exercise $\tau$, we obtain that \eqref{eq:probjj} may be written as
\begin{equation}
\sup_\tau \textbf{E}^{X_0=x}\left[ e^{- r\tau} \left( \textbf{E}^{\tau=t}\left[  \int_t^\infty e^{-r(\tau-s) }\pi(X_s)\ ds  \right] -\delta K\right) 1_{\{\tau<\infty\}} \right].
\label{eq:prob2}
\end{equation}

Let's focus now on the inner expected value $\textbf{E}^{\tau=t}\left[  \int_t^\infty e^{-(\tau-s) }\pi(X_s) \ ds  \right]$. Changing the integration variable follows
\begin{equation}
\textbf{E}^{\tau=t}\left[  \int_0^\infty e^{-rv }\pi(X_{v+\tau})\ dv  \right].
\label{eq:e1}
\end{equation}

Considering $r-\mu>0$, we have that $ \int_0^\infty \int_\Omega    e^{-rv }\pi(X_{v+\tau}) \ dv < \infty$. Since $e^{-rv }\pi(X_{v+\tau})$ is a continuous funtion it's also a measurable function. By both conditions we obtain that it is $[0,\infty) \times \Omega$-integrable. Therefore by the Dominated Convergence Theorem we can interchange the integrals, from which follows by \eqref{eq:e1}, that
\begin{equation}
\int_0^\infty\textbf{E}^{\tau=t}\left[   e^{-rv }\pi(X_{v+\tau}) \right]\ dv
= (\theta-\alpha K)K \int_0^\infty\textbf{E}^{\tau=t}\left[   e^{-rv } X_{v+\tau} \right]\ dv,
\label{eq:e2}
\end{equation}
where we took into account the expression of the profit function $\pi$.


Let's now focus on the expected value $\textbf{E}^{\tau=t}\left[   e^{-rv }  X_{v} \right]$.
It follows that
\begin{align}
\textbf{E}^{\tau=t}\left[   e^{-rv } X_{v+\tau} \right] 
&= \textbf{E}^{\tau=t}\left[   X_\tau e^{\left(\mu- \frac{\sigma^2}{2}-r \right) (\tau+v-\tau) + \sigma (W_{\tau+v}-W_\tau)}\right] \nonumber \\
&=x_\tau e^{\left(\mu- \frac{\sigma^2}{2}-r \right) v} \ \textbf{E}^{\tau=t}\left[ e^{\sigma W_v} \right] \nonumber \\
&= x_\tau e^{\left(\mu- \frac{\sigma^2}{2}-r \right) v} e^{ \frac{\sigma^2}{2} v} \nonumber \\
&=x_\tau e^{(\mu-r)v}.
\label{eq:e4}
\end{align}


In the first step we used the expression associated to the GBM and the fact that knowing the investment time $\tau$, we also know the demand level at that time, here represented as $X_\tau=x_\tau$. In the second step, the fact that the Brownian Motion has stationary increments implies that $W_{\tau+v}-W_\tau \overset{d}{=} W_v -W_0 = W_v$, since we assumed \textbf{W} to be a standard Brownian Motion which implies $W_0=0$.
In the third step we used the fact that $ W_v \sim \mathcal{N}(0,v)$ and the expression for the moment generating function associated to the Normal distribution, from which follows $\textbf{E}\left[e^{sW_v}\right]=e^{\frac{1}{2} s v^2}$. Simplifying the expression we obtain \eqref{eq:e4}.

Plugging the resultant expression \eqref{eq:e4} in \eqref{eq:e2} and solving it, we obtain the formula of the terminal cost function associated to this problem - corresponding to the expression between parenthesis in \eqref{eq:prob2}. We will denote it by $h$ and its expression corresponds to
\begin{equation}
h(x)=\frac{(\theta-\alpha K)K x}{r-\mu}- \delta K.
\label{prob1:h}
\end{equation}
The terminal cost function $h$ represents the discounted long-term profit by acquiring a product when the demand level is $x$. As one can note, it already includes the investment cost of such decision.
 
Denoting $F$ as the value function associated to this problem, we obtain that our optimization problem, as described in \eqref{eq:probjj}, can be written as a standard optimal stopping problem with null running cost function, given by
\begin{equation}
F(x)=\sup_\tau \textbf{E}^{X_0=x}\left[e^{-r\tau } h(X_\tau) \textbf{1}_{\{\tau<\infty\}} \right]=\sup_\tau \textbf{E}^{X_0=x}\left[e^{-r\tau } \left( \frac{(\theta-\alpha K)K X_\tau}{r-\mu}-\delta K \right) \textbf{1}_{\{\tau<\infty\}} \right].
\label{eq:prob3}
\end{equation}



Recurring to Bellman principle, we have that the solution $F$ verifies the variational inequality given by Hamilton-Jacobi-Bellman (HJB) equation
\begin{equation}
\min \{ -rF(x)+ \mathcal{L}F(x), h(x)-F(x) \} =0.
\label{eq:HJB}
\end{equation}
Recalling what was discussed in the previous chapter, we have that the value function is given by
$$F(x)=\begin{cases} a x^{d_1}  \ , \ x \in \mathcal{C} \\
h(x) \ , \ x \in \mathcal{S}
\end{cases},$$
where coefficient $a$ and the threshold value $x^*$, that defines the boundary between continuation and stopping regions, are found by value matching and smooth pasting conditions, expressed by the corresponding system
\begin{equation}
\begin{cases} a(x^*)^{d_1}=\frac{K(\theta-\alpha K) x^*}{r-\mu} - \delta K \\
ad_1(x^*)^{d_1-1}=\frac{K(\theta-\alpha K)}{r-\mu}
\end{cases}
\hspace{5mm} \Rightarrow \ \hspace{5mm}
\begin{cases}
a= \left( \frac{K(\theta-\alpha K) x^*}{r-\mu} - \delta K \right)(x^*)^{-d_1} = \frac{\delta K (x^*)^{-d_1}}{d_1-1}\\
x^*=\frac{d_1}{d_1-1} \frac{ \delta (r-\mu)}{\theta-\alpha K}
\end{cases}
\label{eq:sistema}
\end{equation}
with $d_1=\frac{1}{2}-\frac{\mu}{\sigma^2} +\sqrt{\left( \frac{1}{2} -\frac{\mu}{\sigma^2} \right) ^2+ \frac{2r}{\sigma^2}}>1$, being the positive root of the polynomial described in \eqref{intro:pol}.

It follows the continuation and stopping regions associated to this problem are given by
$$\mathcal{C}=\left\{ x \in \textbf{R}^+: x \leq x^* = \frac{d_1}{d_1-1} \frac{ \delta (r-\mu)}{\theta-\alpha K} \right\}$$
$$\mathcal{S}=\textbf{R}^+ \setminus \mathcal{C}= \left\{ x \in \textbf{R}^+: x > x^* = \frac{d_1}{d_1-1} \frac{ \delta (r-\mu)}{\theta-\alpha K} \right\}.$$

Finally, we are now in the position to formally define the stopping time $\tau$, used in \eqref{eq:prob3}, as $\tau=\inf\{t\geq0: X(t) \in \mathcal{S} \}=\inf\{t\geq 0: X(t) \geq x^* \}$.


%%%%%%%%%%%%%%%%%%%%%%%%%%%%%%%%%%%%%%%%%%%%%%%%%%%%%%%%%%%%%%%%%%%%%%%%%%%%%%%%%%%%


\subsection{Capacity Optimization Model}
\label{subsec:1_com}

Now we increase the complexity, by requiring that the production capacity is chosen to be the maximizer of the discounted long-term profit minus the investment cost, while keeping the goal of finding the best time to invest in the product.

Our problem can be written now as
\begin{equation}
\sup_\tau \textbf{E}^{X_0=x} \left[ \max_K \left\{ e^{-r\tau }  \left( \int_\tau^\infty e^{-r(\tau-s)} \pi(X_s)\ ds -\delta K \right) \right\} \textbf{1}_{\{\tau<\infty\}} \right].
\label{eq:probj}
\end{equation}

Manipulating the expression as done in the previous section \ref{subsec:1_bm}, we obtain that \eqref{eq:probj} may be written as
\begin{equation}
\sup_\tau \textbf{E}^{X_0=x} \left[ e^{-r\tau } \max_K \left\{ h(X_\tau,K) \right\} \textbf{1}_{\{\tau<\infty\}} \right],
\label{eq:q1}
\end{equation}
with $h$ corresponding to the terminal function deduced in \eqref{prob1:h}, in which we now highlight, not only the dependence on the demand level, but also on the production capacity $K$ chosen at the investing time.

In this section, the capacity optimization model is obtained in two steps. In the first one we calculate the capacity level that optimizes the terminal cost function $h$, which we will denote by $K^*$. The second step consists in solving the optimal stopping problem given by $\sup_\tau \textbf{E}^{X_0=x}\left[e^{-r\tau}h(X_\tau,K^*) \textbf{1}_{\{\tau<\infty\}} \right]$, in which we are already considering the optimized terminal function.

The optimal capacity level $K^*$ is found by analyzing the behaviour - namely its concavity and stationary points - of the terminal function $h$, by considering a fix level of demand.

The stationary points are found by calculating the roots of the first partial derivative.
We obtain that the  first partial derivative is given by
$$\frac{\partial h }{\partial K}(x,K)=  \frac{(\theta-2\alpha K)x}{r-\mu} - \delta, $$
which implies that $h$ has a unique stationary point
\begin{equation}
K=\frac{\theta}{2\alpha}-\frac{\delta (r-\mu)}{2 \alpha x}. 
\label{eq:K41}
\end{equation}

We obtain that its second partial derivative is negative and given by
$$\frac{\partial^2 h }{\partial K^2}(x,K)=  -\frac{2\alpha x}{r-\mu}<0, $$
since we assumed $\alpha>0$ and $r-\mu>0$ and the GBM doesn't take negative values. Therefore, $h$ is a concave function and the capacity value found in \eqref{eq:K41} is its global maximizer.

From now on, and to emphasize its maximizer role, we denote \eqref{eq:K41} by $K^*$.
Note that $K^*$ is dependent of the demand level in the sense that the optimal capacity is increasing with the initial observed demand value.

Now we proceed to the second step. Evaluating $h$ at its optimal capacity level $K^*$ we obtain
$$h(x,K^*)=\frac{(\theta x -\delta (r-\mu))^2}{4 \alpha (r-\mu) x}.$$

Denoting $F^*$ as the value function associated to the optimal stopping problem in \eqref{eq:41}, the optimization problem can be stated as
\begin{equation}
F^*(x)=\sup_\tau \textbf{E}^{X_0=x}\left[ e^{-r\tau}h(X_\tau,K^*) \textbf{1}_{\{\tau<\infty\}} \right]
= \sup_\tau \textbf{E}^{X_0=x}\left[ e^{-r\tau} \frac{(\theta X_\tau -\delta (r-\mu))^2}{4 \alpha (r-\mu) X_\tau} \textbf{1}_{\{\tau<\infty\}}\right],
\label{eq:41}
\end{equation}

which is again a standard optimal stopping problem with null running cost functio. Similarly to the benchmark model, we obtain that the value function associated to \eqref{eq:41}, satisfies the HJB variational inequality as described in \eqref{eq:HJB} - although now considering the solution $F^*$ instead of $F$. Therefore $F^*$ is such that

$$F^*(x)=\begin{cases} b x^{d_1}  \ , \ x \in \mathcal{C} \\
h(x,K^*) \ , \ x \in \mathcal{S}
\end{cases},$$
where coefficient $b$ and the threshold value $x_C^*$, that defines the boundary between continuation and stopping regions, are found by value matching and smooth pasting conditions, expressed by the corresponding system
\begin{equation}
\begin{cases} b (x_C^*)^{d_1}=\frac{(\theta x -\delta (r-\mu))^2}{4 \alpha (r-\mu) x} \\
b d_1(x_C^*)^{d_1-1}=\frac{\theta^2 (x_C^*)^2 -\delta^2 (r-\mu)^2}{4 \alpha (r-\mu) (x_C^*)^2}
\end{cases}
\label{eq:sistema3}
\end{equation}
with $d_1=\frac{1}{2}-\frac{\mu}{\sigma^2} +\sqrt{\left( \frac{1}{2} -\frac{\mu}{\sigma^2} \right) ^2+ \frac{2r}{\sigma^2}}>1$, being the positive root of the polynomial described in \eqref{intro:pol}.

We get two possible positive roots for the threshold level: $x^*_{C,1}=\frac{d_1+1}{d_1-1} \frac{ \delta (r-\mu)}{\theta-\alpha K}$ and $x^*_{C,2}=\frac{\delta  (r-\mu )}{\theta }$. However, after some manipulation, we exclude the second one $x^*_{C,2}$, since the coefficient $b$ associated to it takes a null value. This is an absurd, since it would lead to a null value function for any demand level smaller than $x^*_{C,2}$, contradicting the fact that the possibility of investing in the future is also valuable. Therefore we obtain that the threshold level and coefficient $b$ in \eqref{eq:sistema} are, respectively, given by
\begin{align}
 &x_C^*=\frac{d_1+1}{d_1-1} \frac{ \delta (r-\mu)}{\theta} \\
 &b=\left( \frac{(\theta x -\delta (r-\mu))^2}{4 \alpha (r-\mu) x_C^*} \right)(x_C^*)^{-d_1} = \frac{\delta \theta}{\alpha (d_1^2-1)} \left( \frac{d_1+1}{d_1-1} \frac{ \delta (r-\mu)}{\theta} \right)^{-d_1} \nonumber
 \label{prob1_xC}
\end{align}





%%%%%%%%%%%%%%%%%%%%%%%%%%%%%%%%%%%%%%%%%%%%%%%%%%%%%%%%%%%%%%%%%%%%%%%%%%%%%%%%%%%


\section{Comparative Statics}

\subsection{Benchmark Model}
In this section we study the behaviour of the decision threshold $x^*_B$ \eqref{eq:sistema} and $x^*_{C}$ \eqref{prob1_xC} and $K^*$ as described in \eqref{label}, with
the different parameters.

Comparisons between the benchmark and capacity optimization models will be made.\\
\textbf{Proposition:}
Decision threshold $x^*_B$ increases with $r, \sigma, \ K, \ \alpha, \ \delta$ and decreases with $\mu, \theta$.


\textbf{Proof:}

Before showing the main results stated, observe that
$$\phi:=\sqrt{\frac{4 \mu ^2}{\sigma ^4}-\frac{4 \mu }{\sigma ^2}+\frac{8 r}{\sigma ^2}+1}>0
\label{phi}.$$
This is a recurrent expression in most comparative statics sections.

The fact that $\phi>0$ comes from analyzing the expression inside the square root to infere about $\phi$. Since it only has imaginary roots regarding parameter $\mu$ and $\frac{\partial^2}{\partial \mu^2} \left( \frac{4 \mu ^2}{\sigma ^4}-\frac{4 \mu }{\sigma ^2}+\frac{8 r}{\sigma ^2}+1 \right) =\frac{8}{\sigma^2}>0$, it follows that it is always positive, so it is $\phi$.

Now we are in position to explain stated results.

Regarding $r$, we obtain
$$\frac{\partial x^*_B}{\partial r}=\frac{\delta  \left(-(d1-1) d1 \sigma ^2 \phi-2 \mu +2 r \right)}{(d1-1)^2 \sigma ^2 (\alpha  k-\theta ) \sqrt{\frac{4 \mu ^2}{\sigma ^4}-\phi}}>0,$$
Note that its denominator is positive, giving the constraints of our problem.
Analyzing the numerator, we found that it has no possible roots on problem domain. Thus, taking $\mu<0$, we obtain that $-(d1-1) d1 \sigma ^2 \phi-2 \mu +2 r>0 \quad \forall r, \ \sigma$ which implies that the numerator is positive for any parameter values taken into the problem domain.

Regarding $\sigma$, we obtain
$$\frac{\partial x^*_B}{\partial \sigma}=-\frac{2 \delta  (\mu -r) \left(-2 \mu ^2+\mu  \sigma ^2 \left(\phi+1\right)-2 r \sigma ^2\right)}{(d_1-1)^2 \sigma ^5 (\alpha  k-\theta ) \phi}>0.$$
Note that its denominator is negative, since $\alpha-\theta K<0$.
Analyzing the numerator, we found that it has no possible roots on problem domain. Thus, since $-2 \delta  (\mu -r) \left(-2 \mu ^2+\mu  \sigma ^2 \left(\phi+1\right)-2 r \sigma ^2\right) |_{\mu=0}>0$, it follows that the numerator is negative.

Regarding $K$, we obtain
$$\frac{\partial x^*_B}{\partial K}=\frac{\alpha  \delta  d_1 (r-\mu )}{(d_1-1) (\theta -\alpha  k)^2}>0.$$


Regarding $\alpha$ and $\delta$, we obtain
\begin{align*}
\frac{\partial x^*_B}{\partial \alpha}&=\frac{\delta  d_1 k (r-\mu )}{(d_1-1) (\theta -\alpha  k)^2}>0\\
\frac{\partial x^*_B}{\partial \delta}&=\frac{d_1 (r-\mu )}{(d_1-1) (\theta -\alpha  k)}>0,
\end{align*}
from which the result holds.

%Regarding $\delta$, we obtain
%$$\frac{\partial x^*_B}{\partial \delta}=\frac{d_1dd1} (r-\mu )}{(d_1dd1}-1) (\theta -\alpha  k)}>0.$$

Regarding $\mu$, we obtain
$$\frac{\partial x^*_B}{\partial \mu}=\frac{\delta  \left( \left( (d_1-1) d_1 \sigma ^4 \phi -2 \mu ^2+\mu  \sigma ^2 \phi+1 \right)+r \left(2 \mu -\sigma ^2 \left(\phi+1\right)\right)\right)}{(d_1-1)^2 \sigma ^4 (\alpha  k-\theta ) \phi}<0.$$
Observe that its denominator is negative. On the other side, taking into account the our restrictions, we obtain that there are no possible roots for the numerator. Then, evaluating for any plausible parameter values we obtain a positive value, from which follows that the numerator is always positive.

Regarding $\theta$, we obtain
$$\frac{\partial x^*_B}{\partial \sigma}=-\frac{\delta  d_1 (r-\mu )}{(d_1-1) (\theta -\alpha  k)^2}<0,$$
from which the result follows.
 
 
\begin{flushright}
 $\square$
\end{flushright}


\subsection{Capacity Optimization Model}

\textbf{Proposition:}
Decision threshold $x^*_C$ increases with $\sigma,\ \delta$, decreases with $\theta$ and has a monotonic behaviour with  $\mu$. None of any other parameters have effect on $x^*_C$.


\textbf{Proof:}
Regarding $\sigma$, we obtain
$$\frac{\partial x^*_C}{\partial \sigma}=\frac{4 \delta  (\mu -r) \left(-2 \mu ^2+\mu  \sigma ^2 (1+\phi)-2 r \sigma ^2\right)}{(d_1-1)^2 \theta  \sigma ^5 \phi}.$$
Note that its denominator is positive.
Analyzing the numerator, we found that it has no possible roots on problem domain and that it's positive for all parameters.

Regarding $\delta$, we obtain
$$\frac{\partial x^*_C}{\partial \delta}=\frac{(d_1+1) (r-\mu )}{(d_1-1) \theta }>0.$$

Regarding $\mu$, we obtain
$$\frac{\partial x^*_C}{\partial \mu}=\frac{\delta}{(d_1-1)^2 \theta} \left( 1-d_1^2 +(r-\mu)\left(-1+\frac{2\mu-\sigma^2}{\phi \sigma^2} \right) \right)= \begin{cases}
<0 \quad \text{for} \ \mu<\frac{\sigma^2}{2}\\
>0 \quad \text{for} \ \mu>\frac{\sigma^2}{2}
\end{cases}.$$


Regarding $\theta$, we obtain
$$\frac{\partial x^*_C}{\partial \theta}=-\frac{\delta  (d_1+1) (r-\mu )}{(d_1-1) \theta^2}$$

\begin{flushright}
 $\square$
\end{flushright}


To illustrate results above mentioned we performed some numerical illustrations, using software \textit{Mathematica} and its function \texttt{Manipulate}. However here are only able to present static plots - we leave to the interested ones, to see the results achieved with \texttt{Manipulate}.

Unless it is written the opposite, following values were considered:


\begin{table*}[!htb]
	\centering
	%\label{my-label}
	\begin{tabular}{lllllll}
		 $\bullet$ & $\mu=0.03$     &  & \hspace{7cm} &  &  $\bullet$ & $\alpha=0.01$ \\
		 $\bullet$ & $\sigma=0.005$ &  & \hspace{7cm} &  &  $\bullet$ & $\theta=10$   \\
		 $\bullet$ & $r=0.05$       &  & \hspace{7cm} &  &  $\bullet$ & $K=100$       \\
		 $\bullet$ & $\delta=2$                                
	\end{tabular}
%\caption{bjde}
\end{table*}

%\begin{itemize}
%		\item $\mu=0.03$
%		\item $\sigma=0.005$ 
%		\item $r=0.05$
%		\item $\delta=2$
%		\item $\alpha=0.01$
%		\item $\theta=10$
%		\item $K=100$	
%\end{itemize}


We start by illustrating how does $x^*_B$ and $x^*_C$ are related by the capacity level $K$, on which $x^*_B$ is dependent. One can see on Figure \ref{fig:Kvar} that conclusions mentioned on the proof (including that $x^*_B(0)=x^*_C$) hold.

\begin{figure}[!htb]
	\centering
	\includegraphics[width=0.45\textwidth]{Prob1_CapOpt/xopt_k.pdf}
	\caption{Threshold value with respect to the benchmark model (blue) and the capacity optimized model (orange), considering capacity levels $K \in [0, \theta/\alpha)$ and the value that $x^*_B$ takes when considering $K*$ (black).}
	\label{fig:Kvar}
\end{figure}


On Figure \ref{fig:sigm} we observe that both thresholds increase with volatility. This in accordance with \cite{rita} and \cite{hagspiel:cap}, whose works describe that when uncertainty is high, there is a delay time to invest, which is here reflected on an higher demand level.

As the sensibility parameter $\delta$ increases, the firm needs to pay more sunk costs. Therefore the investment will only be made if higher demand values are observed.

\begin{figure}[!htb]
	\begin{subfigmatrix}{2}
		\subfigure[$\sigma \in (0.0001,1)$]{\includegraphics[width=0.45\textwidth]{Prob1_CapOpt/xopt_sigma.pdf}}
		\subfigure[$\delta \in (0,10)$]{\includegraphics[width=0.45\textwidth]{Prob1_CapOpt/xopt_delta.pdf}}
	\end{subfigmatrix}
			\caption{Threshold value with respect to the benchmark model (blue) and the capacity optimized model (orange) and regarding its increasing parameters $\sigma$ and $\delta$.}
			\label{fig:sigm}
\end{figure}

Regarding the drift parameter $\mu$ we obtained that the threshold values do not have a monotonic behaviour, either for smaller or bigger values of volatility. As showed in Figure \ref{fig:mu}, the smallest value of demand level necessary to invest is observed at the stationary point when $\mu=\sigma^2/2$.

\begin{figure}[!htb]
	\begin{subfigmatrix}{2}
			\subfigure[$\sigma=0.05$]{\includegraphics[width=0.45\textwidth]{Prob1_CapOpt/xopt_mu.pdf}}
		\subfigure[$\sigma=0.2$]{\includegraphics[width=0.45\textwidth]{Prob1_CapOpt/xopt_mu_sigma2.pdf}}
	\end{subfigmatrix}
	\caption{Threshold value with respect to the benchmark model (blue) and the capacity optimized model (orange), considering drift $\mu \in [-r, r]$ and corresponding stationary point $\sigma^2/2$ (black).}
	\label{fig:mu}
\end{figure}

On Figure \ref{fig:td} we observe the behaviour of both threshold levels regarding the two other parameters, sensibility level $\delta$ and innovation level $\theta$. We have that the threshold levels increase with $\delta$ and decrease with $\theta$, as previously deduced.

\begin{figure}[!htb]
	\begin{subfigmatrix}{2}
	\subfigure[$r \in ( \mu, 1)$ ]{\includegraphics[width=0.45\textwidth]{Prob1_CapOpt/xopt_r.pdf}}
	\subfigure[$\theta \in (\alpha K=1, 50)$]{\includegraphics[width=0.45\textwidth]{Prob1_CapOpt/xopt_theta.pdf}}
	\end{subfigmatrix}
\caption{Threshold value with respect to the benchmark model (orange) and the capacity optimized model (blue), regarding sensibility parameter $\delta$ and innovation level $\theta$.}
\label{fig:td}
\end{figure}





Now we analyse optimal capacity level $K^*_C$, that is given by evaluating $K^*$ as defined in \eqref{eq:K41} on demand level $x^*_C$, as done in \cite{huis:cap}. Its expression is given by
$$K^*_C=\frac{2 \sigma ^2 \theta}{\alpha \left(\sigma ^2 \left(\sqrt{\frac{4 \mu ^2}{\sigma ^4}-\frac{4 \mu }{\sigma ^2}+\frac{8 r}{\sigma ^2}+1}+3\right)-2 \mu \right)}.$$

\textbf{Proposition:}
Optimal capacity level $K^*_C$ increases with $\mu$, $\sigma$ and $\theta$, decreases with $r$ and $\alpha$ and as no relation with $\delta$.

\textbf{Proof:}
The relation between $K^*_C$ and $\theta$, $r$ or $\alpha$ comes immediately by observing $K^*_C$ expression.

Now, regarding drift parameter we obtain that
 \begin{align*}
\frac{\partial K^*_C(\mu)}{\partial \mu}=
\frac{4 \theta \left(\sigma ^2 \left(\phi+1\right)-2 \mu \right)}{\alpha \phi \left(\sigma ^2 \left(\phi+3\right)-2 \mu \right)^2}>0.
\end{align*}
Since
\begin{align}
\label{cond2}
\sigma ^2 \left(\sqrt{\frac{4 \mu ^2}{\sigma ^4}-\frac{4 \mu }{\sigma ^2}+\frac{8 r}{\sigma ^2}+1}+1\right)-2 \mu>0 
& \Leftrightarrow
\frac{4 \mu ^2}{\sigma ^4}-\frac{4 \mu }{\sigma ^2}+\frac{8 r}{\sigma ^2}+1 > \left( \frac{2 \mu}{\sigma^4}-1 \right)^2=\frac{4 \mu ^2}{\sigma ^4}-\frac{4 \mu }{\sigma ^2}+1 \\
& \Leftrightarrow
\frac{8 r}{\sigma ^2}>0, \nonumber
\end{align}
which is true for $\forall r> 0$ and \eqref{phi} we obtain that both denominator and numerator are positive, from which the result comes.

Regarding volatility parameter we obtain that
 $$    \frac{\partial K^*_C(\sigma)}{\partial \sigma}= 
\frac{8 \theta \left(2 \mu ^2-\mu  \sigma ^2 \left(\phi+1\right)+2 r \sigma ^2\right)}{\alpha \sigma  \phi \left(\sigma ^2 \left(\phi}+3\right)-2 \mu \right)^2}>0$$

One can easily note that the denominator is positive.
When it comes to the numerator, we will study the sign of the expression between parenthesis.
\begin{align}
2 \mu ^2-\mu  \sigma ^2 \left(\phi+1\right)+2 r \sigma ^2 >0 & \Leftrightarrow \left( \frac{2 \mu^2+2r \sigma^2}{\mu \sigma^2} -1 \right)^2 > \frac{4 \mu^2}{\sigma^2}-\frac{4 \mu}{\sigma^2}+\frac{8r}{\sigma^2}+1\\
& \Leftrightarrow r>\mu,
\end{align}
which always hold, implying that the denominator is always positive.

%From \eqref{demo} we obtain that the denominator is positive and from \eqref{condd1} and \eqref{cond2} that the denominator is positive for $\forall r\geq0$, from which the result holds.
\begin{flushright}
	$\square$
\end{flushright}


Considering some numerical approximations, we observe, on Figure \ref{fig:k1}, that $K^*_C$ increases with drift, volatility and innovation level, as deduced before. Note that, regarding the drift parameter, the growth is barely noticeable for negative values of $\mu$, but then it turns to be logarithmic. FINANCIAL INTERPRETATION? This seems to be related with the fact that for small drift values, the future expected demand value is smaller that for positive drift values. Recall that the demand process evolves accordingly to a GBM and its expected value at time $t$ is given by $\textbf{E} ^{X_0=x_0} [X_t]=x_0 e^{\mu t}$.

\begin{figure}[!htb]
	\begin{subfigmatrix}{3}
		\subfigure[$\mu \in ( -r,r )$ ]{\includegraphics[width=0.32\textwidth]{Prob1_CapOpt/kopt_mu.pdf}}
		\subfigure[$\sigma \in (0,1)$]{\includegraphics[width=0.32\textwidth]{Prob1_CapOpt/kopt_sigma.pdf}}
		\subfigure[$\theta \in (\alpha K^*_C,10)$]{\includegraphics[width=0.32\textwidth]{Prob1_CapOpt/kopt_theta.pdf}}
	\end{subfigmatrix}
	\caption{Optimal capacity regarding the threshold value $x^*_C$ and increasing parameters $\mu$, $\sigma$ and $\theta$.}
	\label{fig:k1}
\end{figure}

Regarding discount rate $r$ and sensibility parameter $\alpha$, we have on Figure \ref{fig:k2} that $K^*_C$ decreases with them, as expected.
 
\begin{figure}[!htb]
	\begin{subfigmatrix}{2}
		\subfigure[$ r \in ( \mu, 1 )$]{\includegraphics[width=0.45\textwidth]{Prob1_CapOpt/kopt_r.pdf}}
		\subfigure[$ \alpha \in (0,1)$]{\includegraphics[width=0.45\textwidth]{Prob1_CapOpt/kopt_alpha.pdf}}
	\end{subfigmatrix}
	\caption{Optimal capacity regarding the threshold value $x^*_C$ and decreasing parameters $r$ and $\alpha$.}
	\label{fig:k2}
\end{figure}









\subsection{R\&D and Capacity Optimization Model}
\label{subsec:RDcap1}

PASSOU PARA UM NOVO CAPÍTULO

As stated in section \ref{prob1max}, we are not able to solve analytically the polynomial presented in \eqref{} for every value $\gamma \in (0,1)$. However we considered some numerical approximations, using software \textit{Mathematica} and its function \texttt{Solve}. For the effect, we considered 
\begin{itemize}
	\item $r=0.05$;
	\item $F(X)=10$;
	\item $\gamma \in (0,1]$ incremented by 0.05.
\end{itemize}

Following results are implemented on script \texttt{RVopt.nb}.


\begin{figure}[!htb]
	\centering
	\includegraphics[width=0.6\textwidth]{Prob1_MaxProb/RVlambda_opt05.PNG}
	\caption{Optimal values of $R$ and $V(X)$ for fixed values of $F$ and $r$}
\end{figure}

We obtain that, although the optimal investment $R$ grows with exponent $\gamma$, the value function for the respective optimal $R$ decreases with exponent $\gamma$ (and in a different way from the decreasing of $\lambda(R)$). We get that the smaller value of $V(X)$ is approximately 4.05, corresponding to the optimal investment level $R=1$ and $\lambda(R)=0.45$ and the biggest value of $V(X)$ is approximately 4.69, corresponding to the optimal investment level $R=0.014$ and $\lambda(R)=0.05$